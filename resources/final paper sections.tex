\documentclass{article}
\usepackage[margin=1.0in]{geometry}
\usepackage{authblk} %package for blocking authors, followed by blocking affiliation
\usepackage{url}
\usepackage{ulem} % when using ulem package, must change \emph to \it for italics.
\usepackage[colorlinks,citecolor=red,urlcolor=red,linkcolor=black]{hyperref} %hyperref 
\usepackage{array}
\usepackage{amssymb,amsmath,tabu}
\usepackage{hyperref}
\setlength\parindent{0pt}
\usepackage[english]{babel}
\usepackage{color}


\begin{document}
\title{Guideline for Sections of Research Proposal}
\author[*]{}
\date{}
\maketitle

Remember, as stated in the syllabus, your final research proposal must be in ASA format. This means that you are using 12-point font, and using 1-inch margins (on all sides). Your final paper should have four required sections, in the following order: Introduction, Literature Review, Data \& Methods, and Timeline. 
\\ \\ 
If you include a Reference section, it must be the final section (beginning on a new page after the Timeline) and must also be in ASA format. 
\\ \\
Note, if you are including a section on the background of your case (for those of you in SOC 348), you can include this information in a Context section that follows the Data \& Methods, but comes before the Timeline.
\\ \\
The sections are outlined below:



\section*{Introduction:}
This is where you're introducing the reader to your research question (the relationship between your variables), identifying how you think that relationship works, and providing a brief background on that relationship.

\begin{itemize}
\item General background statement about the relationship between the variables (4 pts)
\newline ``{\it{{\color{red}Social movements engage in various sorts of action to achieve their goals.}}}''
\item An example (or two) about how that relationship works (4 pts)
\newline ``{\it{{\color{red}For example, social movements can lobby politicians or engage in petition drives to impact policy change.}}}''
\item A coherent research question, which can be rephrased, and an reason why that relationship works (4 pts)
\newline ``{\it{{\color{red}Protest may influence the likelihood of changing educational policy because it raises politicians' awareness of the issue and pressures them to respond with concessions.}}}''
	\begin{itemize}
	\item For Social Movements students, you may also relate this question to a specific movement 
	\newline ``{\it{{\color{red}The Occupy movement is an exemplary case to study this relationship because they engaged in various tactics, including protest, to influence educational policy.}}}''
	\end{itemize}
\item Related to this course (4 pts)
	\begin{itemize}
	\item For Research Methods students, this means discussing a relationship between variables that exist in the GSS data set. 
	\item For Social Movements students, this means discussion a relationship between variables that fall under a topic covered in the course. 
	\end{itemize}
\item Coherent/intelligible writing style (4 pts)
\end{itemize}



\section*{Literature Review:}
This is where you're introducing the reader to how the relationship between your variables operates (e.g. how does it work in real life, why is there a relationship between the two variables, why it makes sense that these variables are related). To do this, you need to find academic literature (research articles) that discuss your two variables/concepts, and discuss their findings.

\begin{itemize}
\item General statement about how academic literature demonstrates the relationship between the variables (4 pts)
\newline ``{\it{{\color{red}Scholars have demonstrated the relationship between social movement activity and policy change.}}}''
\item Discussion of one or more academic article's findings, as it relates to your variables (4 pts)
\newline ``{\it{{\color{red}For example, Amenta (1989) shows that things like lobbying or maintaining connections to politicians had a positive effect on the passage of Old-Age policy in the United States. In addition, other work points to protest as being a more effective mobilizer of policy change (Piven and Cloward 1977; McAdam and Su 2002). McAdam and Su (2002) show that disruptive protest increased politicians attention to anti-war issues, which led to more Congressional anti-war legislation.}}}''
\item Summary statement that synthesizes the literature and relates to your research question (4 pts)
\newline ``{\it{{\color{red}This work demonstrates that not only do movement tactics influence policy change, but when those tactics are disruptive, they are more likely to lead policy changes that are beneficial to the movement and its constituents. For these reasons, I believe that college student protest operates in a similar way for educational policy.}}}''
\item Correct ASA format for citations (4 pts)
	\begin{itemize}
	\item \href{https://owl.english.purdue.edu/owl/resource/583/02/}{Formatting Guide}
	\end{itemize}
\item Coherent/intelligible writing style (4 pts)
\end{itemize}

\section*{Data \& Methods:}
This is where you're introducing the reader to how you plan to answer your research question. You will discuss the data you propose to use: either interviews, ethnography, participant observation, or using one or more quantitative data sets. Next, you must discuss your method or analytic strategy, which is your way of analyzing the data. If you're using qualitative data (e.g. interviews, ethnographic field notes, participant observation), analysis usually entails ``grounded theory,'' where code/categorizing each interview, note, or observation into themes (themes that emerge as you're coding/categorizing, not that you've predetermined prior to data collection). If you're using a quantitative data set, analysis usually entails conducting bivariate (e.g. correlation, chi-square, t-test, ANOVA) or multivariate (e.g. linear/multiple/OLS regression, negative binomial/Poisson regression) tests.

\begin{itemize}
\item Characteristics of the data (5 pts)
	\begin{itemize}
	\item Quantitative Data:
	\newline ``{\it{{\color{red}Given my interest in how tactics influence policy change, I use data from the General Social Survey administered in 2014. The data set has 2,538 observations, with individuals as the unit of analysis. The data set includes an array of information, including demographics, behavioral, and attitudinal measures.}}}''
	\item Qualitative Data:
	\newline ``{\it{{\color{red}Given my interest in how tactics influence policy change, I propose to conduct ninety interviews with activists from thirty on-campus, education-based organizations. This will entail interviewing 3 activists from each organization (the organization's leader/president, a member of their executive committee, and a rank-and-file member). }}}''
	\end{itemize}

\item Variables of Interest (5 pts)
	\begin{itemize}
	\item Quantitative Data:
	\newline ``{\it{{\color{red}The data set includes numerous variables to examine the relationship between tactics and policy change. I will use measures related to participation in social movement action (e.g. CIVDIS, OPPSEGOV, etc.) and those related to an individuals beliefs that they can have an impact on government (e.g. EFFGOV, GOVLIST, etc.). In order to get a broad picture of the effect of social movement participation and policy change, I will include demographic information about the participant (e.g. age, sex, race/ethnicity), and I will account for their political orientation and organizational involvement.}}}''
	\item Qualitative Data:
	\newline ``{\it{{\color{red}The interviews will cover several topics to address why the organizations did or did not get involved campaigns to change educational policy. For those organizations that did get involved, I will ask participants about the types of action they engaged in (protest, lobbying, sit-ins, etc.), how they participated (alone or in a coalition), how the tactic changed over time, the immediate impact of their action, whether or not they plan to continue action, and what types of outcomes they wish to initiate.}}}''
	\end{itemize}

\item Method of Analysis (5 pts)
	\begin{itemize}
	\item Quantitative Data:
	\newline ``{\it{{\color{red}I will use regression analysis to understand the effect of protest participation on the belief in policy change. Preliminary bivariate analysis, using correlation, shows a strong positive and significant correlation between measures of movement action/participation (e.g. OPPSEGOV) and the belief that one can impact government policy (e.g. EFFGOV), $r$ = .68, $p$ $<$ .001 {\color{black}[or for ANOVA, an example would be: $F$ = 10.92, $p$ $<$ .001. Or for Chi Square, an example would be: $X^2$ = 9.84, $p$ $<$ .001]}. This demonstrates that as participation increases, so too does belief that you can impact policy change. The regression analysis will build off this relationship by incorporating control variables (e.g. demographics) to account for spuriousness and build a causal model.}}}''
	\item Qualitative Data:
	\newline ``{\it{{\color{red}I will use a grounded theory approach to the study of social movement action and its effect on policy change. I take a grounded approach because it is data-centric and allows me to pull out themes in the data (interviews) as they are collected and analyzed. Not having a predetermined set of themes or categories is beneficial because I will not be wed to categories or codes that don't necessarily appear in the responses of individuals. Moreover, the grounded theory approach allows for the expansion or contraction of categories, based on the themes generated in the interviews. I will compare these themes across interviews, and aim to create theory from themes that cut across responses of activist. Doing so will help paint a broad picture of how activists engage in action to impact policy change.}}}''
	\end{itemize}
\item Coherent/intelligible writing style (5 pts)
\end{itemize}

\section*{Timeline:}
You need to present a timeline to show the audience that you have a plan to get all of this research done.

\begin{itemize}
\item Qualitative Data: {\color{red}
	\begin{itemize} 
	\item December 2016-March 2017:
		\begin{itemize}
		\item Conduct Interviews
		\end{itemize}
	\item April 2017-July 2017:
		\begin{itemize}
		\item Analyze Interviews
		\item Write Up Results
		\end{itemize}
	\item August 2017-October 2017:
		\begin{itemize}
		\item Write Up Results
		\item Submit to Conferences
		\item Present at ASA Conference
		\end{itemize}
	\item November 2017-December 2017:
		\begin{itemize}
		\item Incorporate Feedback
		\item Submit Manuscript for Journal Publication
		\end{itemize}
	\end{itemize} }
\item Quantitative Data: {\color{red}
	\begin{itemize}
	\item December 2016-March 2017:
		\begin{itemize}
		\item Download Data
		\item Clean Data
		\item Run Analyses
		\end{itemize}
	\item April 2017-July 2017:
		\begin{itemize}
		\item Write Up Results
		\item Revise Results
		\end{itemize}
	\item August 2017-October 2017:
		\begin{itemize}
		\item Submit to Conferences
		\item Present at ASA Conference
		\item Incorporate Feedback
		\item Submit Manuscript for Journal Publication
		\end{itemize}
	\end{itemize} }
\end{itemize}
\end{document}











