\documentclass{article}
\usepackage[margin=1.0in]{geometry}
\usepackage{authblk} %package for blocking authors, followed by blocking affiliation
\usepackage{url}
\usepackage{ulem} % when using ulem package, must change \emph to \it for italics.
\usepackage{hyperref}
\usepackage{array}
\usepackage{amssymb,amsmath,tabu}
\usepackage[usenames, dvipsnames]{color}
\usepackage{hyperref}
\usepackage[super]{nth}
\setlength\parindent{0pt}
\usepackage[english]{babel}


\begin{document}
\title{Attention to Movements: Group/Individual Presentation and Paper\\ SOCI 348: Social Movements (Fall 2017)}
\author[*]{}
\date{}
\maketitle

The goal of this project is to demonstrate how ``coverage'' of social movements (and their issues) varies across different sources. For example, some news sources cover a social movement in a positive or neutral manner, others paint the movement negatively. In addition, some sources choose to focus on the ``substance'' of the movement's issue (called substantive coverage) while others focus on non-substantive aspects such as scandals, infighting/factions, and characteristics of the movement's events. Think about these differences as you collect sources. \newline

\begin{centering}
\section*{Presentation (20 points)\\Due: 11/27 -- 12/6}
\end{centering}

In a group of up to 5 people, or individually, students will select a contemporary (or historical) social movement to follow, with the task of gathering news media coverage of that social movement (including newspaper coverage, social media coverage, and televised news coverage, youtube videos). Students will be required to collect no fewer than ten (10) different sources that mention the movement (e.g. 10+ newspaper articles, 10+ twitter accounts, 10+ facebook groups, 10+ news media/youtube videos, or combinations of the above to meet a minimum of ten sources -- the combination is the preferred way to go). However, students should use at least two newspaper articles as sources. \newline

Students will be required to present the results of their findings in a group (or individual) presentation to the class, near the end of the course. In the presentation, students should discuss differences in the types of coverage (negative versus positive) the organization received, and a comparison of the difference across different news media sources. Students' central focus should be on how they would advise the organization on how to increase the attention to the movement and how they could increase positive publicity of their goals (AKA ``substantive'' coverage). The presentation should: \\

\begin{itemize}
\item Present the movement and/or social movement organization
\item Discuss the movement's issues and goals, as well as how many members it has (try to find out that information)
\item Discuss how many of each type of source you analyzed (e.g. 5 newspapers, 4 twitter accounts, 3 videos). 
\item Provide justification for \textit{why} you chose those sources over other sources (e.g. ``it was the first long article that showed up'' or ``the twitter handle frequently appeared when the movement/SMO was discussed'')
\item Discuss patterns and discrepancies \textit{within} each type of source. Is it positive, negative, or neutral? What is covered (e.g. the movement's goals, or something else regarding the movement)?
\item Discuss patterns and discrepancies \textit{between} the different types of sources. How do the types of sources vary (e.g. are newspapers more negative than twitter feeds?) What is covered (e.g. the movement's goals, or something else regarding the movement)? Overall, across all coverage sources, is the coverage generally positive, negative, or neutral?
\item Discuss how you think the movement could do better at getting ``substantive'' coverage.
\end{itemize}

These presentations will take place across four days: Monday, 11/27; Wednesday, 11/29; Monday, 12/4; Wednesday, 12/6. \newline

\begin{centering}
\section*{Final Paper (30 points)\\Due: 12/11 @ 4:20pm}
\end{centering}

In a final paper, due the date and time of our final exam, students will complete a 3-5 page paper (double- or single-spaced, 8$\frac{1}{2}$ inch x 11 inch, 1 inch margins) relating to their ``Attention to Movements'' presentation. The paper should describe the movement and it's goals, discuss the overall coverage it received across all media as well as how that coverage varied between news media sources (highlighting the potential for bias in some sources), and finally provide some recommendations for how the movement could improve it's attention-getting strategy. This paper is due by 4:20pm on Monday, 12/11, the day of our final exam. No late papers will be accepted. \newline

\end{document}











