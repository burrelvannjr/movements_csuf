\documentclass{article}
\usepackage[margin=1.0in]{geometry}
\usepackage{authblk} %package for blocking authors, followed by blocking affiliation
\usepackage{url}
\usepackage{ulem} % when using ulem package, must change \emph to \it for italics.
\usepackage{hyperref}
\usepackage{array}
\usepackage{amssymb,amsmath,tabu}
\usepackage[usenames, dvipsnames]{color}
\usepackage{hyperref}
\usepackage[super]{nth}
\setlength\parindent{0pt}
\usepackage[english]{babel}


\begin{document}
\title{Building a Movement ``How-To'': Individual Reflection Paper\\ SOCI 348: Social Movements (Fall 2017) \\ {\Large{30 points}} \\ {\Large{Due: October 27, 2017 @ 5pm}}}
\author[*]{}
\date{}
\maketitle

In the ``How To'' guide we created as a class (the {\color{blue}\href{https://docs.google.com/a/uci.edu/document/d/1H5a3ikUP_A4Dbw0-Ge_srnCb-MGhcEnaY_yIQENpq6A/edit?usp=sharing}{Google Doc}}), students were asked to think of themselves as seasoned activists working as consultants for a new/emergent social movement organization. The idea was that, if the movement actually existed, we could take that document (and our expertise) to the organization to help the group ``get off the ground.''\newline

Students were asked to create an organization and a name for the group, along with debating the grievances, goals, structure and leadership, how the group would garner resources and gain participants, the organization's strategy and tactics (e.g. protest versus boycott), and how the organization would respond to sanctions, repression, or counter-mobilization. \newline


\section*{Reflection Paper Instructions}

Students are required to write a 1-2 page paper (double- or single-spaced, 8$\frac{1}{2}$ inch x 11 inch, 1 inch margins), reflecting on the ``Building a Movement'' class project. Students should reflect on how they view the movement playing out, and address questions such as: Do you envision the movement achieving its goals? Why or why not? What are some possible responses the movement could encounter? What are some internal problems the movement could face? The paper may also address questions not listed here.\newline

The reflection paper is due by 5pm (online) on Friday, October 27, 2017. No late papers will be accepted.

\end{document}











