\documentclass{article}
\usepackage[margin=1.0in]{geometry} %change all margins to 1.0 inches (except the title, but moves up)
%\documentclass[12pt]{article}
%\usepackage[margin=.8in]{geometry} 
%\usepackage{authblk} %package for blocking authors, followed by blocking affiliation
\usepackage{url}
\usepackage{enumerate}
\usepackage{ulem} % when using ulem package, must change \emph to \it for italics.
\usepackage{hyperref} %allow for hyper links
\usepackage [english]{babel}
\usepackage [autostyle, english = american]{csquotes}
\usepackage[usenames, dvipsnames]{color}
%\definecolor{mypink1}{rgb}{0.858, 0.188, 0.478}
%\definecolor{mypink2}{RGB}{219, 48, 122}
%\definecolor{fully_o}{cmyk}{0, 60, 100, 0}
%\definecolor{mygray}{gray}{0.6}
\MakeOuterQuote{"}
\usepackage[T1]{fontenc} %add encoding for small caps
\def\changemargin#1#2{\list{}{\rightmargin#2\leftmargin#1}\item[]}
\let\endchangemargin=\endlist 
\usepackage{titling} %title margin editing
\setlength{\droptitle}{-.75in} %size of top margin
\usepackage{setspace}
\usepackage{changepage} %changes margins using adjustwidth
\usepackage{tabularx}
\usepackage{tabu}
\usepackage{longtable}
\usepackage[super]{nth}
\usepackage{paralist} %to use compact item stuff
\makeatletter
\newcommand\tabfill[1]{%
\dimen@\linewidth%
\advance\dimen@\@totalleftmargin%
\advance\dimen@-\dimen\@curtab%
\parbox[t]\dimen@{\raggedright #1\ifhmode\strut\fi}%
}
%%%%below changes footer to special footer with name and page number
\usepackage{fancyhdr}
\pagestyle{fancy} %can be {fancy}
\cfoot{\thepage}
\renewcommand{\headrulewidth}{0pt}
%%%%above changes footer to special footer with name and page number
\begin{document}
%\date{today}
%\maketitle

\begingroup  
  \centering
  \begin{spacing}{1.5} %begins 1.5 spacing
  \textsc{\textbf{\LARGE{Social Movements \& Collective Behavior}}} %textsc is small caps, textbf is bold font, huge is largest font possible
  \end{spacing}
  \begin{spacing}{1.0} %begins single-spacing
  \centerline{\large Sociology 348}
  \centerline{\large Fall 2017}
  %\centerline{\normalsize bvann@uci.edu \textbullet \space (714) 398-5815 \textbullet \space \href{http://www.burrelvannjr.com}{burrelvannjr.com}}
  \end{spacing}
\endgroup
\raggedright %left-justifies text AKA does not justify all of text


%\begingroup %date group start
  %��\centerline{} %line space
  %\centerline{} %line space
   %\centerline{( {\it{\today}} )} %today's date, italicized with parentheses
%\endgroup %end date group

%\begin{singlespace}
Time: \textbf{M\&W: 1:00pm--2:15pm} \hfill  \hfill Instructor: \textbf{Burrel Vann Jr} \\
Room: \textbf{H--521} \hfill  \hfill Email: \textbf{\href{mailto:bjvann@fullerton.edu?subject=SOCI\%20348}{bjvann@fullerton.edu}} \\
Website: \textbf{\href{https://moodle-2017-2018.fullerton.edu/course/view.php?id=31395}{SOCI 348}} \hfill  \hfill Office: \textbf{CP--929} \\
  \hfill  \hfill Office Hours: \textbf{M\&W: 3:00pm--4:00pm} \\
%\end{singlespace}





%\begin{singlespace}
\section*{Course Description}
Why do people organize to change the world around them? Why do people sometimes protest but more often not? How do movements work, and why do some succeed while others fail? \newline

At various points in history, social movements have had a substantial impact on social policy and politics. In the 1910s, women organized to earn their right to vote. In the 1960s, many protested the gross injustices endured by African-Americans in defense of civil rights. In the 1970s, college students, draft-dodgers, and others worked together to oppose the Vietnam War. In the late 2000s, a Tea Party movement emerged to protest the government's involvement in the redistribution of wealth to ameliorate income inequality. As these examples show, social movements are critical to American democracy. \newline

This course explores the origins, dynamics, and consequences of social movements. Over the semester, we will examine a wide range of topics including: the emergence of movements, recruitment, interactions between social movements and the general public or political officials, tactics, and the factors contributing to the success and failure of movements.
%%\end{singlespace}

%\begin{singlespace}
\section*{Course Objectives}
\begin{itemize}
\item To introduce students to the concept of social movements, and how they differ from other forms of collective behavior.\vspace*{-.75em}
\item To examine the emergence, development, dynamics and tactics, and impacts of social movements in the United States.\vspace*{-.75em}
\item To gain a familiarity with current research on social movements.\vspace*{-.75em}
\item To gain learn by putting movement theories into practice.
\end{itemize}

\textbf{CSUF Sociology Student Learning Outcomes (SLOs)} 
\begin{enumerate}
\item Students will apply key sociological concepts.\vspace*{-.75em}
\item Students will compare, contrast, and critique major theoretical and epistemological orientations in sociology including functionalism, conflict, interactionism, and feminism.\vspace*{-.75em}
\item Students will demonstrate critical thinking from various sociological perspectives, such as reflecting on their social location, evaluating the implicit assumptions of everyday life, challenging commonsense understandings, and assessing the structure of an argument.\vspace*{-.75em}
\item Students will demonstrate clear and effective written and oral communication skills.\vspace*{-.75em}
\item Students will demonstrate knowledge of qualitative and quantitative research design and methods and evaluate their appropriate use.\vspace*{-.75em}
\item Students will use sociological knowledge and skills to engage with local and global communities for the purpose of social justice.\vspace*{-.75em}
\item Students will demonstrate a critical understanding of power, privilege, and oppression across a range of cultures, human experiences, and the intersections of social locations and historical experiences, including their own.
\end{enumerate}

\section*{Required Materials}
\textbf{Textbook} \newline
David A. Snow and Sarah A. Soule (2010). \textit{A Primer on Social Movements}. New York, NY: W.W. Norton \& Company, Inc.

\section*{Course Requirements}
In order to engage in a lively discussion of social movement topics, students are required to complete the weekly readings (chapters and/or articles). In addition, each student must lead at least two article discussions (in a group of 2--4 people), complete 1 short research question quiz, 3 ``refresher'' quizzes on course material, complete a class-wide project on movement-building, and a group project on the attention to movement issues. \newline

\textbf{Participation: Attendance and Discussion on Readings (50 points)} [{\color{blue}SLOs: 1, 3, 4, 5, 6, 7}] \newline
Attendance for this class is not mandatory, but on-time attendance is critical for your overall success in the course. If you miss a class meeting, look on the course website for material you may have missed. Second, if you find it difficult to understand some of the material, get in contact with your one or more of your classmates. Third, if you still find it difficult, set aside time to meet with me in office hours. If my office hours don't work, email me so that we can schedule a time to meet. I reserve the right to re-do a lecture. \newline

In addition, students should participate in discussions on the weekly readings. Each weekly reading is geared toward helping you grasp of concepts about social movement emergence, dynamics and tactics, and impacts. Each week, students will have to complete a set of between one and four readings, which will consist of a combination of readings from the required text and articles. All articles will be provided on Titanium. \newline

%When reading articles, it's best to read them quickly. To do so, you don't need to read the entire article. Instead, first read the abstract, then the introduction, then conclusion. (If you so desire, you can go back to read the literature review and the data/methods section, noting section headings and tables/graphs. This is only important if you are trying to learn more about how the researcher collected their data, or if you want to explore some of the literature the researchers used.)\newline

\textbf{Discussion Leader Participation (50 points, 25 points each)} [{\color{blue}SLOs: 1, 3, 4, 5, 6, 7}]\newline
During the first week, students will be asked to sign up for at least two articles for which they will be discussion leaders. The discussions for each article will be led by at least two students who will develop no fewer than 5 discussion questions related to their article, which will be used as the springboard for discussion. (The discussion leaders will need to meet prior to their discussion day to discuss the assigned materials and organize their questions and comments.)  The discussion should focus on key themes that run across the material, emphasizing theoretical insights and claims.  \newline

\textbf{Concept Quizzes (30 points, 5 points each)} [{\color{blue}SLOs: 1, 2, 4, 6}] \newline
Over the course of the semester, students will be asked to complete a series of six (6) quizzes to assess their understanding of key concepts in the study of social movements. These quizzes will be given at random, at the start of class, and will cover material from chapters in the book (e.g. not articles). Quizzes will cover concepts taken from chapter readings done in the previous weeks. Concept quizzes will begin at the start of class, last for 5-10 minutes and will be immediately returned. There will be no makeup concept quizzes given. \newline

\textbf{Research Question Quiz (20 points)} [{\color{blue}SLOs: 1, 4, 5, 6}]\newline
During the semester, students are required to complete one (1) research question quiz related to a social movements topic covered in this course that interests the student. This quiz is designed to help you develop a strong research question for future potential research on social movements. This quiz will also serve to get students thinking about what topics/subjects to emphasize for our class project on ``building a movement.'' Therefore, this online quiz is due by 5pm on Friday, 10/13. \newline

\textbf{``Building a Movement'': Class Project (20 points)} [{\color{blue}SLOs: 1, 3, 4, 5, 6, 7}] \newline
Toward the middle of the semester, once we've completed a majority of the readings, students will be asked to engage in a unified class project. Students will be asked to, as one, create a hypothetical social movement organization -- such as a ``Homes for Our Own'' group that is dedicated to fighting for affordable housing for the homeless in Fullerton. As such, students will be asked to devise a name for the group, the grievances, goals, and structure and leadership for the group. Students will discuss how the group would garner resources and gain participants. Moreover, students will debate the movement's strategy and tactics (e.g. protest versus boycott) for achieving it's goals. Students will also have to discuss how they would respond to sanctions, repression, or counter-mobilization. \newline

For this assignment, students should see themselves as a seasoned activist working as a consultant for a new/emergent social movement organization. As such, all students will be participating in a unified Google Document where students can add their thoughts in bullet-point fashion. The document should be considered a ``How To'' guide for the hypothetical movement. That is, if the movement actually existed, we could take that document (and our expertise) to the organization to help the group ``get off the ground.'' We will work on this class project on Monday, 10/16 and Wednesday, 10/18. \newline


\textbf{``Building a Movement'': Reflection Paper (30 points)} [{\color{blue}SLOs: 1, 3, 4, 5, 6, 7}]\newline
Students will write a 1-2 page reflection on the ``building a movement'' class project, focusing particularly on the ``How To'' guide/Google Document created by the class. Students should reflect on how they view the movement playing out, and address questions such as: Do you envision the movement achieving its goals? Why or why not?(Hint: most of the time, the answer is no... most movements fail to achieve their stated goals... but achieve some combination of success and failure). What are some possible responses the movement could encounter? What are some internal problems the movement could face? This reflection paper will be due by 5pm on Friday, 10/27. No late reflection papers will be accepted. \newline


\textbf{``Attention to Movements'': Group or Individual Presentation (20 points)} [{\color{blue}SLOs: 1, 3, 4, 5, 6, 7}] \newline
In a group of up to 5 people, or individually, students will select a contemporary (or historical) social movement to follow, with the task of gathering news media coverage of that social movement (including newspaper coverage, social media coverage, and televised news coverage). Students will be required to collect no fewer than ten (10) different sources that mention the movement (e.g. 10+ newspaper articles, 10+ twitter accounts, 10+ facebook groups, 10+ news media videos, or any combination of the above to meet a minimum of ten sources -- the combination is the preferred way to go). \newline

Students will be required to present the results of their findings in a group (or individual) presentation to the class, near the end of the course. In the presentation, students should discuss differences in the types of coverage (negative versus positive) the organization received, and a comparison of the difference across different news media sources. Students' central focus should be on how they would advise the organization on how to increase the attention to the movement and how they could increase positive publicity of their goals. These presentations will take place across four days: Monday, 11/27; Wednesday, 11/29; Monday, 12/4; Wednesday, 12/6. \newline

\textbf{``Attention to Movements'': Group or Individual Paper (30 points)} [{\color{blue}SLOs: 1, 2, 3, 4, 5, 6, 7}] \newline
In a final paper, due the date and time of our final exam, students will complete a 3-5 page paper relating to their ``Attention to Movements'' presentation. The paper should describe the movement and it's goals, discuss the overall coverage it received across all media as well as how that coverage varied between news media sources (highlighting the potential for bias in some sources), and finally provide some recommendations for how the movement could improve it's attention-getting strategy. This paper is due by 4:20pm on Monday, 12/11, the day of our final exam. No late papers will be accepted. \newline


\textbf{Extra Credit}\newline
Students may be given the opportunity to complete one extra credit presentation on how a contemporary social movement relates to a topic covered in the course, worth a maximum of 15 points. I reserve the right to provide an extra credit assignment.

\section*{Grading Breakdown}
Final grades will be based on participation in class discussions (50 points), two articles as discussion leader (25 points each), one research question quiz (20 points), six social movement concept quizzes (30 points), participation in ``building a movement'' class project (20 points each), a reflection paper on the ``building a movement'' class project (30 points), a group or individual presentation on the attention to a contemporary social movement (20 points), and a final paper written in a group or individually that clarifies the findings of the ``movement attention'' presentation (30 points) for a total of 250 points. A +/- grading system will not be used. 

\begin{tabbing}
\quad \quad \quad \= Participation: Attendance and Discussion on Readings \quad \quad \quad \quad \= \tabfill{50}\\
\> Discussion Leader Participation \> \tabfill{50 (2, 25 points each)}\\
\> Concept Quizzes \> \tabfill{30 (6, 5 points each)}\\
\> Research Question Quiz \> \tabfill{20}\\
\> ``Building A Movement'': Participation (Class Project) \> \tabfill{20}\\
\> ``Building A Movement'': Reflection Paper (Individual) \> \tabfill{30}\\
\> ``Attention to Movements'': Presentation (Group or Individual) \> \tabfill{20}\\
\> ``Attention to Movements'': Paper (Group or Individual) \> \tabfill{30}\\
\> Total  \> \tabfill{250}
\end{tabbing}

\textbf{Letter Grades}
\vspace*{-.5em}
\begin{tabbing}
\quad \quad \quad \= A = 90\% and above \\
\> B = 80\% and above \\
\> C = 70\% and above \\
\> D = 60\% and above \\
\> F = Below 60\% \\
\end{tabbing}

\section*{Classroom Conduct}
Please be courteous to your classmates and me by remaining engaged and respectful. Students are expected to conduct themselves in a way that does not interfere with the educational experience of others. Additionally, turn cell phones and other electronic devices on silent during class time. Laptops may be used for taking notes or running analyses while in class.

\section*{Academic Integrity}
The California State University, Fullerton policy on academic integrity is explained in {\color{blue}\href{http://www.fullerton.edu/senate/publications_policies_resolutions/ups/UPS%20300/UPS%20300.021.pdf}{University Policy Statement 300.021}}. All work you turn in, including homework assignments, exams, and quizzes must be your own. At the discretion of the instructor, any student found to have engaged in academic dishonesty (including but not limited to plagiarism and/or cheating) will be subject to disciplinary action at the course-level (including but not limited to oral reprimand; ``F'' or ``0'' on the assignment; grade reduction on assignment or course; or ``F'' in the course) or university-level (including but not limited to a report to the student(s) involved, to the department chair, and to the Dean of Students office, Student Conduct, the alleged incident of academic dishonesty, including relevant documentation, actions taken by the instructor including grade sanction, and recommendations for additional action that he/she deems appropriate). 

\section*{Students with Special Needs}
Please inform the instructor during the first week of classes about any disability or special needs that you may have that may require specific arrangements related to attending class sessions, carrying out class assignments, or writing papers or examinations. According to California State University policy, students with disabilities must document their disabilities at the Disability Support Services (DSS) Office in order to be accommodated in their courses. Additional information can be found at the {\color{blue}\href{http://www.fullerton.edu/dss/}{DSS website}}, by calling 657-278-3112, or by email at {\color{blue}\href{mailto:dsservices@fullerton.edu}{dsservices@fullerton.edu}}.


\section*{Emergency Preparedness}
Information about CSUF's emergency preparedness policy can be found at {\color{blue}\href{http://prepare.fullerton.edu/}{Campus Emergency Preparedness}}.


\section*{Changes to Material}
I reserve the right to make changes to the syllabus, including the course outline, at any time, based on the pace of the class.


\newpage

\section*{Course Schedule}

\subsubsection*{1 - \textit{Introductions; What is a Social Movement?} (8/21 \& 8/23)}
\begin{itemize}
\item \textbf{Chapter(s)}: 1 
\end{itemize}

\vspace{3pt}

\subsubsection*{2 - \textit{Conceptualizing Social Movements} (8/28 \& 8/30)}
\begin{itemize}
\item \textbf{Chapter(s)}: 1 
\end{itemize} 

\vspace{3pt}

\subsubsection*{3 - \textit{Political Context and Opportunities; Resources} ({\color{Orange}9/6})}
\begin{itemize}
\item \textbf{Chapter(s)}: {\color{Orange}3 (pp. 64-87); 3 (pp. 87-98)}
\item \textbf{Article(s)}: {\color{Orange}Almeida (2003)}
\end{itemize}

\vspace{3pt}

\subsubsection*{4 - \textit{Grievances} ({\color{blue}9/11} \& {\color{Orange}9/13})}
\begin{itemize}
\item \textbf{Chapter(s)}: {\color{blue}2}
\item \textbf{Article(s)}: {\color{Orange}Van Dyke and Soule (2002); Snow et al. (2005)}
\end{itemize}

\vspace{3pt}

\subsubsection*{5 - \textit{Organizations} ({\color{blue}9/18} \& {\color{Orange}9/20})}
\begin{itemize}
\item \textbf{Article(s)}: {\color{blue}Robnett (1996); Staggenborg (1988);} {\color{Orange}Taylor (1989); McVeigh et al. (2014)}
\end{itemize}

\vspace{3pt}

\subsubsection*{6 - \textit{Framing and Media} (9/25 \& {\color{Orange}9/27})}
\begin{itemize}
\item \textbf{Article(s)}: {\color{Orange}Snow et al. (2007); Roscigno and Danaher (2001)}
\end{itemize}

\vspace{3pt}


\subsubsection*{7 - \textit{Recruitment, Participation, Identity} ({\color{blue}10/2} \& {\color{Orange}10/4})}
\begin{itemize}
\item \textbf{Chapter(s)}: {\color{blue}4}
\item \textbf{Article(s)}: {\color{Orange}Corrigall-Brown et al. (2009); Einwohner (2006); Futrell and Simi (2004)}
\end{itemize}

\vspace{3pt}

\subsubsection*{8 - \textit{Dynamics and Tactics} ({\color{blue}10/9} \& {\color{Orange}10/11})}
\begin{itemize}
\item \textbf{Chapter(s)}: {\color{blue}5}
\item \textbf{Article(s)}: {\color{Orange}McAdam (1983); Van Dyke (2003)}
\item \textbf{Due}: Research Question Quiz (10/13 by 5pm)
\end{itemize}

\vspace{3pt}

\subsubsection*{9 - \textit{``Building A Movement'': Class Project} (10/16 \& 10/18)}
\begin{itemize}
\item \textbf{Activities}: Participation in discussion of movement-building, contribution to course ``How To'' Google Document
\end{itemize}
\vspace{3pt}

\subsubsection*{10 - \textit{Political Consequences/Outcomes} ({\color{blue}10/23} \& {\color{Orange}10/25})}
\begin{itemize}
\item \textbf{Chapter(s)}: {\color{blue}6 (pp. 202-217)}
\item \textbf{Article(s)}: {\color{Orange}Andrews (2001); Andrews (1997); McAdam and Su (2002)}
\item \textbf{Due}: ``Building A Movement'' Reflection Paper (10/27 by 5pm)
\end{itemize}

\vspace{3pt}

\subsubsection*{11 - \textit{Cultural Consequences/Outcomes} ({\color{blue}10/30} \& {\color{Orange}11/1})}
\begin{itemize}
\item \textbf{Chapter(s)}: {\color{blue}6 (pp. 217-220)}
\item \textbf{Article(s)}: {\color{Orange}Bail (2012); Amenta et al. (2009)}
\end{itemize}

\vspace{3pt}

\subsubsection*{12 - \textit{Historical and Contemporary Movements; ``Attention to Movements'': Group or Individual Projects} (11/6 \& 11/8)}
\begin{itemize}
\item \textbf{Activities}: Exposure to historical and contemporary social movements; students encouraged to sit-in on organizational meetings; Library/gathering data on social movement coverage
\end{itemize}

\vspace{3pt}

\subsubsection*{13 - \textit{``Attention to Movements'': Group or Individual Projects} (11/13 \& 11/15)}
\begin{itemize}
\item \textbf{Activities}: Library/gathering data on social movement coverage
\end{itemize}

\vspace{3pt}


\subsubsection*{14 - NO CLASS: Fall/Thanksgiving Break \newline} 

\vspace{3pt}

\subsubsection*{15 - \textit{``Attention to Movements'': Group or Individual Presentations} (11/27 \& 11/29)\newline}

\vspace{3pt}

\subsubsection*{16 - \textit{``Attention to Movements'': Group or Individual Presentations} (12/4 \& 12/6)\newline}

\vspace{3pt}

\subsubsection*{Finals Week - \textit{``Final Exam''} (12/11, 2:30pm--4:20pm)}
\begin{itemize}
\item \textbf{Due}: ``Attention to Movements'': Group or Individual Paper (12/11 by 4:20pm)
\end{itemize}




\end{document}